\section{Implementation Details}
\label{sec:Implementation}

\subsection{ZeroMQ Messaging Pattern}
Even though \emph{libzmq} has a Publisher-Subscriber pattern available, we decided to use Request-Reply pattern for the following reason: the \emph{ZeroMQ} pub-sub pattern does not allow for replies on a get or put request and does not allow for the control of the message queues, which would make it impossible for us to control the exactly once factor. The reason for this is that the pub-sub pattern does not guarantee the arrival of the messages in certain scenarios, such as, quoting the documentation, "when a network failure occurs, or just if the subscriber or network can’t keep up with the publisher"\cite[Chapter~5]{zeromqguide}. This choice, of course, meant more implementation work, but it was a necessary sacrifice.


\subsection{Messages} 

All the four functions execute similar actions:

\begin{enumerate}
    \item Send the request
    \item Receive the reply
    \item Analyze the reply and take action accordingly
\end{enumerate}

All messages are sent in the form of a serialized \emph{Rust struct} - \textbf{Message}. This serialization step is done using a library called \emph{Bson}\footnote{\url{https://docs.rs/bson/latest/bson/}}. 

\begin{lstlisting}
struct Message {
    msg_type: String,
    payload: Bson
}
\end{lstlisting}

The message struct is composed by a message type and a payload as depicted above. The payload is a serialized struct, also using \emph{Bson}. This serialized struct is of a different type, depending on the type of message. 

\subsubsection{Get}

\begin{lstlisting}
pub struct Request {
    pub sub_id: String,
    pub topic: String
}
pub struct Reply {
    pub sub_id: String,
    pub message_no: u64,
    pub broker_id: String,
    pub payload: Vec<u8>,
}
\end{lstlisting}

The reply message contains the message number so that a confirmation on the correct order of messages can be performed inside the library's functions.

The broker id exists solely to inform the user that the broker could not recover its state in a crash.

Furthermore, on this requests we have also acknowledge requests and replies.
\begin{lstlisting}
pub struct Ack {
    pub sub_id: String,
    pub message_no: u64
}
pub struct AckReply {
    pub sub_id: String,
    pub message_no: u64
}
\end{lstlisting}
The goal of the \emph{Ack} message is to guarantee that a given subscriber does not miss a message for a network failure, as will be described in the \ref{sec:Reliability} section. The \emph{AckReply} only exists due to the obligation of the broker to reply to a request per ZeroMQ Req-Rep pattern.


\subsubsection{Put}

\begin{lstlisting}
pub struct Request {
    pub pub_id: String,
    pub topic: String,
    pub message_uuid: String,
    pub payload: Vec<u8>
}
pub struct Reply {
    pub message_uuid: String,
    pub topic: String,
    pub broker_id: String
}
\end{lstlisting}
The message uuid is used to ensure a given message is not sent twice to the broker. This mechanism will be better explained in the \ref{sec:Reliability} section.


\subsubsection{Subscribe}

\begin{lstlisting}
pub struct Request {
    pub sub_id: String,
    pub topic: String
}
pub struct Reply {
    pub sub_id: String,
    pub topic: String,
    pub broker_id: String,
    pub post_offset: u64
}
    
\end{lstlisting}
The post\_offset represents the current post counter on the topic the subscriber subscribed.
\subsubsection{Unsubscribe}

\begin{lstlisting}
pub struct Request {
    pub sub_id: String
}
pub struct Reply {
    pub sub_id: String,
    pub broker_id: String
}
\end{lstlisting}

\subsubsection{Error}

\begin{lstlisting}
enum BrokerErrorType {
    SubscriberNotRegistered,
    SubscriberAlreadyRegistered,
    InhexistantTopic,
    DuplicateMessage,
    TopicMismatch,
    AckMessageMismatch,
    UnknownMessage,
    NoPostsInTopic,
    NotExpectingAck
}

struct BrokerErrorMessage {
    error_type: BrokerErrorType,
    broker_id: String,
    description: String
}
\end{lstlisting}
The error message is used for when the request made to the broker has generated any problem. These errors can be both internal errors, which will be treated by the library functions, or user errors. In the latter case, the error messages are returned to the user directly through the return value of the library's functions.

\subsection{Publisher and Subscriber Contexts}

In order to make the state of a user well defined, we created two structs that holds such data for each type of user (Subscribers and Publishers):

\begin{lstlisting}
struct PublisherContext {
    pub_id: String,
    known_broker_id: Option<String>,
}

struct SubscriberContext {
    sub_id: String,
    topic: String,
    known_broker_id: Option<String>,
    next_post_no: u64
}
\end{lstlisting}

While the \emph{SubscriberContext} subscriber id and the number of the next post should be saved in the same structure and were the primary motivation for these, the rest of the data is merely to improve user experience and allow for a similar interface for each of the cases, subscriber or publisher.


\subsection{Broker}

The broker process is responsible for storing all the posts coming from publishers and for answering all the requests from subscribers. In order to do so, all the information inherent to the requests must be stored in appropriate data structures.

\begin{lstlisting}

enum SubscriberStatus {
    WaitingAck,
    WaitingGet
}

struct SubscriberData {
    topic: String,
    status: SubscriberStatus,
    last_read_post: u64
}

struct TopicData {
    posts: HashMap<String, Vec<u8>>,
    post_counter: u64
}

struct BrokerState {
    broker_uuid: String,
    subs: HashMap<String, SubscriberData>,
    topics: HashMap<String, TopicData>,
    received_uuids: HashSet<String>,
}
\end{lstlisting}

The Broker's three main data structures, as shown in the pseudo-code above:

\begin{description}
    \item[topics] An \emph{Hashmap} containing information on every topic (posts, number of last published message, etc.). Each value of the \emph{hashmap} is a struct containing another hashmap, which links the number of each post to its payload (the number is in String format because Bson did not allow u64 to be the key for a map when serializing the map to save in a file) 
    \item[subs] An \emph{Hashmap} containing information on every subscriber (topic it is subscribed to, last\_read\_post, etc.)
    \item[received\_uuids] An \emph{Hashset} that stores the unique ids of the publisher's messages, to ensure a message is not sent twice
\end{description}

The usage of \emph{hashmaps} and \emph{hashsets} is justified by the temporal efficiency they bring to the table:
    
\begin{itemize}
    \item You can get to the particular post in compile time using its topic and post number through the two nested \emph{hashmaps} in \emph{TOPICS} and \emph{TopicData.posts}  
    \item You can retrieve data from the desired user using the \emph{hashmap} in constant time
    \item You can check if a UUID already exists in the \emph{hashset} in constant time
\end{itemize}

The processing of the incoming requests is done in a simple manner, as depicted in Fig. \ref{fig:activity-diagram}.

